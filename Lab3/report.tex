\documentclass[11pt,a4paper]{report}

%%% Работа с русским языком
\usepackage{cmap}					% поиск в PDF
\usepackage{mathtext} 				% русские буквы в фомулах
\usepackage[T2A]{fontenc}			% кодировка
\usepackage[utf8]{inputenc}			% кодировка исходного текста
\usepackage[english,russian]{babel}	% локализация и переносы

\usepackage{fancyhdr}

\usepackage{lipsum}
\usepackage{etoolbox}

% Code in Latex
\usepackage{listings}

%%% Работа с картинками
\usepackage{graphicx}  % Для вставки рисунков
%\graphicspath{{images/}}  % папки с картинками
\setlength\fboxsep{3pt} % Отступ рамки \fbox{} от рисунка
\setlength\fboxrule{1pt} % Толщина линий рамки \fbox{}
\usepackage{wrapfig} % Обтекание рисунков и таблиц текстом
\usepackage[export]{adjustbox}

%%% Дополнительная работа с математикой
\usepackage{amsmath,amsfonts,amssymb,amsthm,mathtools} % AMS
\usepackage{icomma} % "Умная" запятая: $0,2$ --- число, $0, 2$ --- перечисление
\usepackage{systeme}

\usepackage{tabularx}
\usepackage{tikz-cd}

\patchcmd{\maketitle}
  {\end{titlepage}}
  {\thispagestyle{titlepagestyle}\end{titlepage}}
  {}{}

\fancypagestyle{titlepagestyle}
{
   \fancyhf{}
   \fancyfoot[C]{3 курс, 3 группа}
   \renewcommand{\headrulewidth}{0 mm}
}

\pagestyle{plain}

\begin{document}
	
\lstset{ 
	language=Python, 
	tabsize=2, 
	showspaces=false, 
	showstringspaces=false, 
	float=[htb], 
	captionpos=b, 
	basicstyle=\footnotesize,
	numberblanklines=false, 
} 



\title{Отчет по лабораторной работе №3}

\author{Снопов П.М.}
\thispagestyle{titlepagestyle}
\maketitle
\begin{center}
	\textbf{Лабораторная работа №3}
	
	Определение меры обусловленности Тодда симметричной матрицы простой структуры с использованием методов прямой и обратной итерации. 
	
	\textit{Вариант 10}
\end{center}

\paragraph{1. Постановка задачи}
Пусть дана симметричная матрица простой структуры $A \in \boldsymbol{M}_{n}(\mathbb{R}) $. Необходимо определить меру обусловленности Тодда. Раз матрица простой структуры, то количество ее собственных значений с учетом кратности равно ее размерности. Мера обусловленности Тодда же определяется как произведение норм матрицы и ее обратной. То есть, в силу симметричности матрицы, меру обусловенности Тодда можно определить как отношение абсолютной величины наибольшего собственного значения к абсолютной величине наименьшего.
\paragraph{2. Метод решения}
Чтобы определить меру обусловленности Тодда, необходимо сначала найти наибольшее и наименьшее по модулю собственное значение. Для этих задач воспользуемся методами прямой и обратной итерации. Метод прямой итерации позволит определить наибольшее по модулю собственное значение.
\subparagraph{Описание степенного метода}
Пусть $x_0$ -- произвольный вектор из $\mathbb{R}_n$. Вычисление собственного значения производится итерационно по следующей схеме:
\[
\left\{\begin{aligned}
&v^{(k)} = \frac{x^{(k)}}{\Vert x^{(k)} \Vert}\\
&x^{(k+1)} = Av^{(k)}\\
&\sigma^{(k)} = v^{(k)^T}x^{(k+1)}\\
&k \in \mathbb{N}
\end{aligned}\right.
\]
В работе Praktische Verfahren der Gleichungsauflösung Рихарда фон Мизеса доказано, что
\[
\left\{\begin{aligned}
&\sigma^{(k)} = \lambda_{max}\\
&v^{(k)} = \pm x_{max}\\
&k \rightarrow \infty
\end{aligned}\right.
\]
Где $x_{max}$ --  собственный вектор, соответствующий собственному значению $\lambda_{max}$
\newline
Теперь применим метод обратной итерации для поиска наибольшего собственного значения для матрицы, обратной данной, которое также будет являться наименьшим собственной значением данной матрицы.
\subparagraph{Описание степенного метода}
Пусть также $x_0$ -- произвольный вектор из $\mathbb{R}_n$. Вычисление собственного значения производится итерационно по следующей схеме:
\[
\left\{\begin{aligned}
&v^{(k)} = \frac{x^{(k)}}{\Vert x^{(k)} \Vert}\\
&x^{(k+1)} = A^{-1}v^{(k)}\\
&\alpha^{(k)} = v^{(k)^T}x^{(k+1)}\\
&k \in \mathbb{N}
\end{aligned}\right.
\]
В статье Berechnung der Eigenschwingungen statisch-bestimmter Fachwerke Эрнста Полхаузена доказана следующая сходимость:
\[
\left\{\begin{aligned}
&\alpha^{(k)} = \frac{1}{\lambda_{min}}\\
&v^{(k)} = \pm x_{min}\\
&k \rightarrow \infty
\end{aligned}\right.
\]
Причем на каждом шаге вектор $x^{(k+1)}$ определяется как решение уравнения $Ax^{(k+1)} = v^{(k)}$.
Тогда число обусловленности Тодда вычисляется по следующей формуле:
\[
\mu(A) = \frac{|\lambda_{max}|}{|\lambda_{min}|}
\]
\paragraph{3. Основные процедуры}
Основные функции, используемые при решении задачи:


\begin{lstlisting}
def power_iteration(A: np.array, num_iterations: np.int64) -> np.double:
\end{lstlisting}
Функция, являющая реализацией метода прямых итераций
\begin{lstlisting}
def inverse_iteration(A: np.array, num_iterations: np.int64) -> np.double:
\end{lstlisting}
Функция, являющая реализацией метода обратных итераций
\paragraph{4. Результаты тестирования}
Для тестирования придется генерировать матрицы с заданными собственными значениями и собственными векторами. Получим собственные значения с помощью равномерного распределения. Для получения собственных векторов, получим вектор из равномерного распределения $\omega$ и найдем матрицу Хаусхолдера $H$:
\[
	H = E - 2\omega\omega^T
\] Далее, для генерации матрицы, воспользуемся спектральным разложением матрицы, т.е. представлением матрицы $A$ в виде произведения $H\Lambda H^-1$, где  $\Lambda$ -- диагональная матрица с соответствующими собственными значениями на главной диагонали. Также будем число итераций $K$ вычислять так: $K:=10N$, где $N$ -- размерность системы. Помимо этого для тестирования будем генерировать различные интервалы, в которых лежат собственные значения. Т.к. \lstinline{np.random.rand} генерирует собственные значения из интервала (0,1), то нужно построить соответствующий гомеоморфизм в необходимые интервалы (a,b). Это можно сделать, например, следующим образом:
\[
	f:(0,1) \rightarrow (a,b): x \mapsto a + (b-a)x
\]  
Так, что следующая диаграмма коммутативна:
\begin{center}
	\begin{tikzcd}
	\mathbb{N} \arrow[r] \arrow[dr, dashrightarrow]
	& (0,1) \arrow[d]\\
	& (a,b)
	\end{tikzcd}
\end{center}
Тестирование представлено в следующей таблице:
\begin{table}[htbp]
	
	\begin{center}
		\caption{Таблица тестирования}
		\small
		\begin{tabularx}{\linewidth}{|X|X|X|X|X|X|}
			\hline
			Размерность системы N & Диапазон значений $\lambda$ & Число итераций $K$ & Ср. отн. точность $\mu$ & Ср. отн. точность $\lambda_{max}$ & Ср. отн. точность $\lambda_{min}$ \\
			\hline
			$10$ & $(0,1)$ & $100$& $0.0621$ & $0.0062$ & $1.3356e^{-8}$\\
			\hline
			$10$ & $(0,10)$ & $100$& $0.0287$ & $0.0322$ & $5.0308e^{-12}$\\
			\hline
			$10$ & $(0,100)$ & $100$& $0.4139$ & $0.9149$ & $5.7731e^{-15}$\\
			\hline
			$100$ & $(0,1)$ & $1000$& $0.0344$ & $0.0002$ & $8.6736e^{-18}$\\
			\hline
			$100$ & $(0,10)$ & $1000$& $0.1031$ & $0.0006$ & $2.3564e^{-11}$\\
			\hline
			$100$ & $(0,100)$ & $1000$& $0.0093$ & $0.0070$ & $0.0002$\\
			\hline
			$1000$ & $(0,1)$ & $10000$& $0.1300$ & $2.7441e^{-5}$ & $9.427e^{-8}$\\
			\hline
			$1000$ & $(0,10)$ & $10000$& $0.0721$ & $0.0003$ & $3.2549e^{-8}$\\
			\hline
			$1000$ & $(0,100)$ & $10000$& $1.3603$ & $0.0136$ & $7.0295e^{-6}$\\
			\hline
		\end{tabularx}	
	\end{center}
\end{table}
\end{document}
