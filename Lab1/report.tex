\documentclass[11pt,a4paper,twoside]{report}

%%% Работа с русским языком
\usepackage{cmap}					% поиск в PDF
\usepackage{mathtext} 				% русские буквы в фомулах
\usepackage[T2A]{fontenc}			% кодировка
\usepackage[utf8]{inputenc}			% кодировка исходного текста
\usepackage[english,russian]{babel}	% локализация и переносы

\usepackage{fancyhdr}

\usepackage{lipsum}
\usepackage{etoolbox}

% Code in Latex
\usepackage{listings}

%%% Дополнительная работа с математикой
\usepackage{amsmath,amsfonts,amssymb,amsthm,mathtools} % AMS
\usepackage{icomma} % "Умная" запятая: $0,2$ --- число, $0, 2$ --- перечисление


\patchcmd{\maketitle}
  {\end{titlepage}}
  {\thispagestyle{titlepagestyle}\end{titlepage}}
  {}{}

\fancypagestyle{titlepagestyle}
{
   \fancyhf{}
   \fancyfoot[C]{3 курс, 3 группа}
   \renewcommand{\headrulewidth}{0 mm}
}

\pagestyle{plain}

\begin{document}

\title{Отчет по лабораторной работе №1}

\author{Снопов П.М.}
\thispagestyle{titlepagestyle}
\maketitle
\begin{center}
	\textbf{Лабораторная работа №1}
	
	Решение разреженных СЛАУ специального вида
	
	\textit{Вариант 10}
\end{center}

\paragraph{1. Постановка задачи}
Необходимо решить разреженную СЛАУ специального вида. Портрет матрицы: 
\[
\begin{pmatrix}
	* & * &  &   &   &   &   &   &   &   \\
	* & * & * &   &   &   &   &   &   &   \\
	* & * & * & * &   &   &   &   &   &   \\
	* & * & * & * & * &   &   &   &   &   \\
	* & * &   & * & * & * &   &   &   &   \\
	* & * &   &   & * & * & * &   &   &   \\
	* & * &   &   &   & * & * & * &   &   \\
	* & * &   &   &   &   & * & * & * &   \\
	* & * &   &   &   &   &   & * & * & * \\
	* & * &   &   &   &   &   &   & * & * 
\end{pmatrix}
\]

\paragraph{2. Метод решения}
Приведем вышеописанную матрицу к следующему виду, т.е. устраним первый
столбец, как в стандартном методе Гаусса решения СЛАУ: 
\[
	\begin{pmatrix}
		1 & * &  &   &   &   &   &   &   &   \\
		0 & * & * &   &   &   &   &   &   &   \\
		0 & * & * & * &   &   &   &   &   &   \\
		0 & * & * & * & * &   &   &   &   &   \\
		0 & * &   & * & * & * &   &   &   &   \\
		0 & * &   &   & * & * & * &   &   &   \\
		0 & * &   &   &   & * & * & * &   &   \\
		0 & * &   &   &   &   & * & * & * &   \\
		0 & * &   &   &   &   &   & * & * & * \\
		0 & * &   &   &   &   &   &   & * & * 
	\end{pmatrix}
\] 
Далее, со второй строкой уже так не выйдет: на второй строке есть
элемент 3-его столбца, который содержит нули. Если повторить процедуру
устранения столбца по методу Гаусса, нулевой элемент станет ненулевым.
Поэтому теперь стоит пойти обратным ходом: \[
\begin{pmatrix}
1 & * &  &  &  &  &  &  &  &  \\
0 & * & * &  &  &  &  &  &  &   \\
0 & * & * & * &  &  &  &  &  &   \\
0 & * & * & * & * &  &  &  &  &  \\
0 & * &  & * & * & * &  &  &  &  \\
0 & * &  &  & * & * & * &  &  &  \\
0 & * &  &  &  & * & * & * &  &  \\
0 & * &  &  &  &  & * & * & * &  \\
0 & * &  &  &  &  &  & * & * & 0 \\
0 & * &  &  &  &  &  &  & * & 1  \\
\end{pmatrix}
\] Продолжая процесс: \[
\begin{pmatrix}
1 & 0 &  &  &  &  &  &  &  &  \\
0 & 1 & 0 &  &  &  &  &  &  &   \\
0 & * & 1 & 0 &  &  &  &  &  &   \\
0 & * & * & 1 & 0 &  &  &  &  &  \\
0 & * &  & * & 1 & 0 &  &  &  &  \\
0 & * &  &  & * & 1 & 0 &  &  &  \\
0 & * &  &  &  & * & 1 & 0 &  &  \\
0 & * &  &  &  &  & * & 1 & 0 &  \\
0 & * &  &  &  &  &  & * & 1 & 0 \\
0 & * &  &  &  &  &  &  & * & 1  \\
\end{pmatrix}
\] И дальше уже завершая процедуру, обнулим оставшиеся значения: \[
\begin{pmatrix}
1 & 0 &  &  &  &  &  &  &  &  \\
0 & 1 & 0 &  &  &  &  &  &  &   \\
0 & 0 & 1 & 0 &  &  &  &  &  &   \\
0 & 0 & 0 & 1 & 0 &  &  &  &  &  \\
0 & 0 &  & 0 & 1 & 0 &  &  &  &  \\
0 & 0 &  &  & 0 & 1 & 0 &  &  &  \\
0 & 0 &  &  &  & 0 & 1 & 0 &  &  \\
0 & 0 &  &  &  &  & 0 & 1 & 0 &  \\
0 & 0 &  &  &  &  &  & 0 & 1 & 0 \\
0 & 0 &  &  &  &  &  &  & 0 & 1  \\
\end{pmatrix}
\]

Сложность алгоритма $\mathcal{O}(n)$ 
\paragraph{3. Основные процедуры}
Основные методы, используемые при решении задачи:


\begin{lstlisting}[language=Python]
def solve(A,b)
\end{lstlisting}
метод решения уравнения вида Ax=b
\begin{lstlisting}[language=Python]
def error(x,x_hat,q)
\end{lstlisting}
метод подсчета средней относительной погрешности системы
\begin{lstlisting}[language=Python]
def accuracy(x, x_hat)
\end{lstlisting}
метод подсчета среднего значения оценки точности
\paragraph{4. Результаты тестирования}

\begin{center}
	\begin{tabular}{|c|p{0.2\linewidth}|p{0.2\linewidth}|p{0.2\linewidth}|p{0.2\linewidth}|}
		\hline
		№ теста & Размерность системы & Диапазон значений элементов матрицы  & Средняя относительная погрешность системы & Среднее значение оценки точности \\
		\hline
		1 & 10 & (-10,10) & 1.432e-15 & 7.748e-15  \\
		\hline
		2 & 10 & (-100, 100) & 1.554e-16 & 1.192e-15 \\
		\hline
		3 & 10 & (-1000, 1000) & 3.886e-16 & 8.450e-15 \\
		\hline
		4 & 100 & (-10, 10) & 5.846e-15 & 3.786e-14 \\
		\hline
		5 & 100 & (-100, 100) & 2.376e-15 & 5.23e-15 \\
		\hline
		6 & 100 & (-1000, 1000) & 2.265e-15 & 6.505e-14 \\
		\hline
		7 & 1000 & (-10, 10) & 4.759e-13 & 3.199e-13 \\
		\hline
		8 & 1000 & (-100, 100) & 7.456e-13 & 7.456e-13 \\
		\hline
		9 & 1000 & (-1000, 1000) & 7.544e-13 & 1.032e-11 \\
		\hline
	\end{tabular}	
\end{center}

\end{document}
