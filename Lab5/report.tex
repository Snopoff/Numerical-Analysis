\documentclass[11pt,a4paper]{report}

%%% Работа с русским языком
\usepackage{cmap}					% поиск в PDF
\usepackage{mathtext} 				% русские буквы в фомулах
\usepackage[T2A]{fontenc}			% кодировка
\usepackage[utf8]{inputenc}			% кодировка исходного текста
\usepackage[english,russian]{babel}	% локализация и переносы

\usepackage{fancyhdr}

\usepackage{lipsum}
\usepackage{etoolbox}

% Code in Latex
\usepackage{listings}

%%% Работа с картинками
\usepackage{graphicx}  % Для вставки рисунков
%\graphicspath{{images/}}  % папки с картинками
\setlength\fboxsep{3pt} % Отступ рамки \fbox{} от рисунка
\setlength\fboxrule{1pt} % Толщина линий рамки \fbox{}
\usepackage{wrapfig} % Обтекание рисунков и таблиц текстом
\usepackage[export]{adjustbox}

%%% Дополнительная работа с математикой
\usepackage{amsmath,amsfonts,amssymb,amsthm,mathtools} % AMS
\usepackage{icomma} % "Умная" запятая: $0,2$ --- число, $0, 2$ --- перечисление
\usepackage{systeme}

\usepackage{tabularx}
\usepackage{tikz-cd}

%%% Работа с картинками
\usepackage{graphicx}  % Для вставки рисунков
%\graphicspath{{images/}}  % папки с картинками
\setlength\fboxsep{3pt} % Отступ рамки \fbox{} от рисунка
\setlength\fboxrule{1pt} % Толщина линий рамки \fbox{}
\usepackage{wrapfig} % Обтекание рисунков и таблиц текстом
\usepackage[export]{adjustbox}

\patchcmd{\maketitle}
  {\end{titlepage}}
  {\thispagestyle{titlepagestyle}\end{titlepage}}
  {}{}

\fancypagestyle{titlepagestyle}
{
   \fancyhf{}
   \fancyfoot[C]{3 курс, 3 группа}
   \renewcommand{\headrulewidth}{0 mm}
}

\pagestyle{plain}

\begin{document}

\lstset{
	language=Python,
	tabsize=2,
	showspaces=false,
	showstringspaces=false,
	float=[htb],
	captionpos=b,
	basicstyle=\scriptsize,
	numberblanklines=false,
	breaklines=true
}



\title{Отчет по лабораторной работе №5}

\author{Снопов П.М.}
\thispagestyle{titlepagestyle}
\maketitle
\begin{center}
	\textbf{Лабораторная работа №5}

	Метод универсальной дифференциальной прогонки для линейных уравнений второго порядка.

	\textit{Вариант 9}
\end{center}

\paragraph{1. Постановка задачи}
Метод универсальной дифференциальной прогонки для линейных уравнений второго порядка:
\newline
Имеется линейное дифференциальное уравнение второго порядка 
\[
	y''(x) + p(x)y'(x) = q(x)y(x) + f(x) \text{, $ x \in [a, b] $}
\]
С краевыми условиями:
\[
\begin{aligned}
& \alpha_0y(a) + \beta_0y'(a) = A \\
& \alpha_1y(b) + \beta_1y'(b) = B
\end{aligned}
\]
Используя метод дифференциальной прогонки, решить данное дифференциальное уравнение
\paragraph{2. Метод решения}
Задача решается в 2 этапа: 
\subparagraph{Прямая прогонка}  Предполагая, что $\beta_0 \neq 0$, решаем 2 независимые друг от друга задачи Коши на $[a,b]$:
\[
	\left\{
		\begin{aligned}
			& z_1' = -z_1^2 - p(x)z_1 + q(x) \\
			& z_1(a) = - \frac{\alpha_0}{\beta_0}
		\end{aligned}
	\right.
	\left\{
		\begin{aligned}
			& z_2' = z_2 [z_1 + p(x)] + f(x) \\
			& z_2(b) = \frac{A}{\beta_0}
		\end{aligned}
	\right.
\]
Если $\beta_0 = 0$, то $\alpha_0 \neq 0 \text{Если, конечно} A \neq 0$, и тогда решаем 2 другие независимые друг от друга задачи Коши на $[a,b]$:
\[
	\left\{
		\begin{aligned}
			& z_1' = -z_1^2q(x) + z_1p(x) + 1 \\
			& z_1(a) = - \frac{\beta_0}{\alpha_0}
		\end{aligned}
	\right.
	\left\{
		\begin{aligned}
			& z_2' = -z_1[z_2q(x) + f(x)] \\
			& z_2(b) = - \frac{A}{\alpha_0}
		\end{aligned}
	\right.
\]
\subparagraph{Обратная прогонка} При обратной прогонке получаем приближенное решение исходной краевой задачи. Если $\beta_0 \neq 0$, то решаем следующую задачу Коши:
\[
	\left\{
		\begin{aligned}
			& y' = z_1y + z_2 \\
			& y(b) = \frac{B - \beta_1z_2(b)}{\alpha_1+\beta_1z_1(b)}
		\end{aligned}
	\right.
\]
Если же $\alpha_0 \neq 0$, то задача Коши имеет вид:
\[
	\left\{
		\begin{aligned}
			& y' = \frac{y - z_2}{z_1} \\
			& y(b) = \frac{Bz_1(b) + \beta_1z_2(b)}{\beta_1+\alpha_1z_1(b)}
		\end{aligned}
	\right.
\]

\paragraph{3. Основные процедуры}
Основные функции, используемые при решении задачи:


\begin{lstlisting}
def RungeKutta(f: Callable, h: float, x: float, y=0) -> float:
\end{lstlisting}
Функция, соответствующая методу Рунге-Кутта 4 порядка(формула Кутта-Менсона)
\begin{lstlisting}
def forward(X: dict, b_cond1: list, interval: list) -> dict,dict:
\end{lstlisting}
Функция, реализующая прямую прогонку
\begin{lstlisting}
def backward(X: dict, b_cond1: list, b_cond2: list, interval: list, Z1: dict, Z2: dict) -> dict:
\end{lstlisting}
Функция, реализующая обратную прогонку
\begin{lstlisting}
def write_data(X:dict, Y:dict, Yprime:dict):
\end{lstlisting}
Функция записывающая полученные данные
\begin{lstlisting}
def solve():
\end{lstlisting}
Основная функция, осуществляющая решение задачи.

\end{document}