\documentclass[11pt,a4paper]{report}

%%% Работа с русским языком
\usepackage{cmap}					% поиск в PDF
\usepackage{mathtext} 				% русские буквы в фомулах
\usepackage[T2A]{fontenc}			% кодировка
\usepackage[utf8]{inputenc}			% кодировка исходного текста
\usepackage[english,russian]{babel}	% локализация и переносы

\usepackage{fancyhdr}

\usepackage{lipsum}
\usepackage{etoolbox}

% Code in Latex
\usepackage{listings}

%%% Работа с картинками
\usepackage{graphicx}  % Для вставки рисунков
%\graphicspath{{images/}}  % папки с картинками
\setlength\fboxsep{3pt} % Отступ рамки \fbox{} от рисунка
\setlength\fboxrule{1pt} % Толщина линий рамки \fbox{}
\usepackage{wrapfig} % Обтекание рисунков и таблиц текстом
\usepackage[export]{adjustbox}

%%% Дополнительная работа с математикой
\usepackage{amsmath,amsfonts,amssymb,amsthm,mathtools} % AMS
\usepackage{icomma} % "Умная" запятая: $0,2$ --- число, $0, 2$ --- перечисление
\usepackage{systeme}

\patchcmd{\maketitle}
  {\end{titlepage}}
  {\thispagestyle{titlepagestyle}\end{titlepage}}
  {}{}

\fancypagestyle{titlepagestyle}
{
   \fancyhf{}
   \fancyfoot[C]{3 курс, 3 группа}
   \renewcommand{\headrulewidth}{0 mm}
}

\pagestyle{plain}

\begin{document}

\title{Отчет по лабораторной работе №2}

\author{Снопов П.М.}
\thispagestyle{titlepagestyle}
\maketitle
\begin{center}
	\textbf{Лабораторная работа №2}
	
	Метод Халецкого для решения переопределенных СЛАУ
	
	\textit{Вариант 10}
\end{center}

\paragraph{1. Постановка задачи}
Необходимо решить переопределенную СЛАУ. Т.е. СЛАУ, у которой число уравнений больше, чем число неизвестных. Пусть $A \in \boldsymbol{M}_{N \times S}(\mathbb{R}) $, где $N>S$. Также определим диагональную матрицу весов $B \in \boldsymbol{M}_{S}(\mathbb{R})$.
\newline
Т.е. нужно решить уравнение $Ax=f$.

\paragraph{2. Метод решения}
Приведем матрицу $A$ к квадратному виду, умножив слева на $A^T$. Тогда получим уравнение $A^TAx=A^Tf$, где $A^TA \in \boldsymbol{M}_{S}(\mathbb{R})$. Так как имеем матрицу весов $B$, то тогда уравнение имеет вид $A^TBAx=A^TBf$. Пусть $\hat{A} = A^TBA$, $\hat{f} = A^TBf$, тогда уравнение имеет вид $\hat{A}x = \hat{f}$, где $\hat{A} \in \boldsymbol{M}_{S}(\mathbb{R})$.
\newline
Заметим, что $\hat{A}$ -- симметричная, положительно-определенная матрица, а значит $\hat{A} = LL^T$, где $L$ -- нижняя треугольная матрица. Тогда составим систему из 2 уравнений:
\[
\left\{\begin{aligned}
	&Ly    = \hat{f}\\
	&L^Tx = \hat{f}
\end{aligned}\right.
\]

%Сложность алгоритма $\mathcal{O}(n)$ 
\paragraph{3. Основные процедуры}
Основные функции, используемые при решении задачи:


\begin{lstlisting}[language=Python]
	def cholesky(A, b, f):
\end{lstlisting}
Функция, которая совершает разложение Холецкого для переопределенной СЛАУ с вектором весовых коэффициентов.
\begin{lstlisting}[language=Python]
	def residual(A, x, f):
\end{lstlisting}
Функция, которая считает вектор невязки и его норму.
\paragraph{4. Результаты тестирования}
Для тестирования возьмем следующую систему уравнений:
\[
\left\{\begin{aligned}
&x = 0\\
&y = 0\\
&2x + y = 8
\end{aligned}\right.
\]
Для следующих весовых коэффициентов: $\{(1,1,1), (2,2,1), (1,1,2)\}$.
\newline
Найдем решение, посчитаем вектор невязки и его норму:
\begin{table}[htbp]
	\begin{center}
		\caption{Таблица тестирования}
		\begin{tabular}{|c|c|c|c|}
			\hline
			Вектор весов & Решение системы & Вектор невязки  & Норма вектора невязки \\
			\hline
			$(1,1,1)$ & $(2.667, 1.333)^T$ & $(2.667, 1.333,-1.333)^T$& 3.266\\
			\hline
			$(2,2,1)$ & $(2.286, 1.143)^T$ & $(2.286, 1.143,-2.286)^T$& 3.429\\
			\hline
			$(1,1,2)$ & $(2.909, 1.455)^T$ & $(2.909, 1.455,-0.727)^T$& 3.333 \\
			\hline
		\end{tabular}	
	\end{center}
\end{table}
Проверим результаты, решив вручную систему:
\newline
\subparagraph{Решение системы для вектора весов $(1,1,1)$}
Диагональная матрица весов является единичной матрицей, поэтому умножение на нее не изменяет коэффциентов системы. Имеем следующее уравнение:
\begin{gather*}
	\begin{pmatrix} 1&0\\0&1\\2&1 \end{pmatrix}\begin{pmatrix} x\\y \end{pmatrix}=\begin{pmatrix} 0\\0\\8 \end{pmatrix}
\end{gather*}
Домножим слева на $\begin{pmatrix} 1&0\\0&1\\2&1 \end{pmatrix}^T$:
\begin{gather*}
	\begin{pmatrix} 1&0&2\\0&1&1 \end{pmatrix}\begin{pmatrix} 1&0\\0&1\\2&1 \end{pmatrix}\begin{pmatrix} x\\y \end{pmatrix}=\begin{pmatrix} 1&0&2\\0&1&1 \end{pmatrix}\begin{pmatrix} 0\\0\\8 \end{pmatrix}
\end{gather*}
Тогда:
\begin{gather*}
	\begin{pmatrix} 5&2\\2&2 \end{pmatrix}\begin{pmatrix} x\\y \end{pmatrix}=\begin{pmatrix} 16\\8 \end{pmatrix}
\end{gather*}
Отсюда:
\begin{gather*}
	\begin{pmatrix} 1&0\\0&1 \end{pmatrix}\begin{pmatrix} x\\y \end{pmatrix}=\begin{pmatrix} \dfrac{8}{3}\\[2ex]\dfrac{4}{3} \end{pmatrix}
\end{gather*}
Т.е. $\tilde{x} = \begin{pmatrix} \dfrac{8}{3}\\[2ex]\dfrac{4}{3} \end{pmatrix}$. Посчитаем теперь $A\tilde{x}$:
\begin{gather*}
	\begin{pmatrix} 1&0\\0&1\\2&1 \end{pmatrix}\begin{pmatrix} \dfrac{8}{3}\\[2ex]\dfrac{4}{3} \end{pmatrix}=\begin{pmatrix} \dfrac{8}{3}\\[2ex]\dfrac{4}{3}\\[2ex]\dfrac{20}{3} \end{pmatrix} =: \tilde{f}
\end{gather*}
Теперь посчитаем вектор невязки $f_r = f - \tilde{f}$ и его (евклидову) норму $f_r = (\sum_{i=1}^{N} f_r(i)^2)^{\frac{1}{2}}$:
\begin{gather*}
	f_r =  \begin{pmatrix} \dfrac{8}{3}\\[2ex]\dfrac{4}{3}\\[2ex]\dfrac{20}{3} \end{pmatrix} - \begin{pmatrix} 0\\0\\8 \end{pmatrix} = \begin{pmatrix} \dfrac{8}{3}\\[2ex]\dfrac{4}{3}\\[2ex]-\dfrac{4}{3} \end{pmatrix}
\end{gather*}
\[
f_r = (\sum_{i=1}^{N} f_r(i)^2)^{\frac{1}{2}} = \frac{4}{3}\sqrt6
\]
\subparagraph{Решение системы для вектора весов $(2,2,1)$}  % ИЗМЕНИТЬ
Имеем следующее уравнение:
\begin{gather*}
\begin{pmatrix} 1&0\\0&1\\2&1 \end{pmatrix}\begin{pmatrix} x\\y \end{pmatrix}=\begin{pmatrix} 0\\0\\8 \end{pmatrix}
\end{gather*}
Домножим слева на $\begin{pmatrix} 1&0\\0&1\\2&1 \end{pmatrix}^T\begin{pmatrix} 2&0&0\\0&2&0\\0&0&1 \end{pmatrix}$:
\begin{gather*}
\begin{pmatrix} 1&0\\0&1\\2&1 \end{pmatrix}^T\begin{pmatrix} 2&0&0\\0&2&0\\0&0&1 \end{pmatrix}\begin{pmatrix} 1&0\\0&1\\2&1 \end{pmatrix}\begin{pmatrix} x\\y \end{pmatrix}=\begin{pmatrix} 1&0\\0&1\\2&1 \end{pmatrix}^T\begin{pmatrix} 2&0&0\\0&2&0\\0&0&1 \end{pmatrix}\begin{pmatrix} 0\\0\\8 \end{pmatrix}
\end{gather*}
Тогда:  
\begin{gather*}
\begin{pmatrix} 6&2\\2&3 \end{pmatrix}\begin{pmatrix} x\\y \end{pmatrix}=\begin{pmatrix} 16\\8 \end{pmatrix}
\end{gather*}
Отсюда:
\begin{gather*}
\begin{pmatrix} 1&0\\0&1 \end{pmatrix}\begin{pmatrix} x\\y \end{pmatrix}=\begin{pmatrix} \dfrac{16}{7}\\[2ex]\dfrac{8}{7} \end{pmatrix}
\end{gather*}
Т.е. $\tilde{x} = \begin{pmatrix} \dfrac{16}{7}\\[2ex]\dfrac{8}{7} \end{pmatrix}$. Посчитаем теперь $A\tilde{x}$:
\begin{gather*}
\begin{pmatrix} 1&0\\0&1\\2&1 \end{pmatrix}\begin{pmatrix} \dfrac{16}{7}\\[2ex]\dfrac{8}{7} \end{pmatrix}=\begin{pmatrix} \dfrac{16}{7}\\[2ex]\dfrac{8}{7}\\[2ex]\dfrac{40}{7} \end{pmatrix} =: \tilde{f}
\end{gather*}
Теперь посчитаем вектор невязки $f_r = f - \tilde{f}$ и его (евклидову) норму $f_r = (\sum_{i=1}^{N} f_r(i)^2)^{\frac{1}{2}}$:
\begin{gather*}
f_r =  \begin{pmatrix} \dfrac{16}{7}\\[2ex]\dfrac{8}{7}\\[2ex]\dfrac{40}{7} \end{pmatrix} - \begin{pmatrix} 0\\0\\8 \end{pmatrix} = \begin{pmatrix} \dfrac{16}{7}\\[2ex]\dfrac{8}{7}\\[2ex]-\dfrac{16}{7} \end{pmatrix}
\end{gather*}
\[
f_r = (\sum_{i=1}^{N} f_r(i)^2)^{\frac{1}{2}} = \frac{24}{7}
\]
\subparagraph{Решение системы для вектора весов $(1,1,2)$} 
Имеем следующее уравнение:
\begin{gather*}
\begin{pmatrix} 1&0\\0&1\\2&1 \end{pmatrix}\begin{pmatrix} x\\y \end{pmatrix}=\begin{pmatrix} 0\\0\\8 \end{pmatrix}
\end{gather*}
Домножим слева на $\begin{pmatrix} 1&0\\0&1\\2&1 \end{pmatrix}^T\begin{pmatrix} 1&0&0\\0&1&0\\0&0&2 \end{pmatrix}$:
\begin{gather*}
\begin{pmatrix} 1&0\\0&1\\2&1 \end{pmatrix}^T\begin{pmatrix} 1&0&0\\0&1&0\\0&0&2 \end{pmatrix}\begin{pmatrix} 1&0\\0&1\\2&1 \end{pmatrix}\begin{pmatrix} x\\y \end{pmatrix}=\begin{pmatrix} 1&0\\0&1\\2&1 \end{pmatrix}^T\begin{pmatrix} 1&0&0\\0&1&0\\0&0&2 \end{pmatrix}\begin{pmatrix} 0\\0\\8 \end{pmatrix}
\end{gather*}
Тогда:  
\begin{gather*}
\begin{pmatrix} 9&4\\4&3 \end{pmatrix}\begin{pmatrix} x\\y \end{pmatrix}=\begin{pmatrix} 32\\16 \end{pmatrix}
\end{gather*}
Отсюда:
\begin{gather*}
\begin{pmatrix} 1&0\\0&1 \end{pmatrix}\begin{pmatrix} x\\y \end{pmatrix}=\begin{pmatrix} \dfrac{32}{11}\\[2ex]\dfrac{16}{11} \end{pmatrix}
\end{gather*}
Т.е. $\tilde{x} = \begin{pmatrix} \dfrac{32}{11}\\[2ex]\dfrac{16}{11} \end{pmatrix}$. Посчитаем теперь $A\tilde{x}$:
\begin{gather*}
\begin{pmatrix} 1&0\\0&1\\2&1 \end{pmatrix}\begin{pmatrix} \dfrac{32}{11}\\[2ex]\dfrac{16}{11} \end{pmatrix}=\begin{pmatrix} \dfrac{32}{11}\\[2ex]\dfrac{16}{11}\\[2ex]\dfrac{80}{11} \end{pmatrix} =: \tilde{f}
\end{gather*}
Теперь посчитаем вектор невязки $f_r = \tilde{f} - f$ и его (евклидову) норму $f_r = (\sum_{i=1}^{N} f_r(i)^2)^{\frac{1}{2}}$:
\begin{gather*}
f_r =  \begin{pmatrix} \dfrac{32}{11}\\[2ex]\dfrac{16}{11}\\[2ex]\dfrac{80}{11} \end{pmatrix} - \begin{pmatrix} 0\\0\\8 \end{pmatrix} = \begin{pmatrix} \dfrac{32}{11}\\[2ex]\dfrac{16}{11}\\[2ex]-\dfrac{8}{11} \end{pmatrix}
\end{gather*}
\[
f_r = (\sum_{i=1}^{N} f_r(i)^2)^{\frac{1}{2}} = \frac{8}{11}\sqrt{21}
\]
\subparagraph{График системы}
\begin{figure}[h!]
	\centering
	\includegraphics[width=\linewidth, keepaspectratio=true]{plot.png}
	\caption{График системы и полученных точек}
	\label{fig:plot}
\end{figure}
\end{document}
